\chapter{Pemrograman Berorientasi Objek}
\section{Gaya dan Bahasa Pemrograman} %15.1

banyak jenis bahasa pemrograman dan hampir banyak juga gaya pemrograman (kadang disebut paradigma). Program-program yang telah kita buat sebelumnya adalah bergaya prosedural, karena menitik beratkan spesifik pada prosedur yang dilakukan komputer.

Kebanyakan program Java adalah berorientasi objek, yang mana berfokus pada objek-objek dan interaksi-interaksi mereka. Berikut adalah karakteristik dari Pemrograman Berorientasi Objek :

\begin{itemize}
	\item{Objek selalu merepresentasikan entitas di dunia nyata. Pada bab sebelumnya, membuat class Deck adalah satu langkah menuju Pemrograman Berorientasi Objek.}
	\item{Kebanyakan dari method-method dalam class Deck adalah method objek(seperti method-method yang anda panggil pada Strings) daripada method class(seperti method-method pada class Math). Method-method yang telah kita tulis sejauh ini adalah method class. Pada bab ini kita akan menulis beberapa method objek.}
	\item{Objek-objek saling terisolasi satu sama lain 	dengan membatasi bagaimana mereka berinteraksi, khususnya dengan mencegah pengaksesan instance variables(attribute objek/variabel instance) tanpa dengan memanggil method.}
	\item{Class teroganisir dalam pohon keluarga dimana class-class baru memperluas(menurunkan sifat) dari class yang telah ada, menambahkan method baru dan menggantikan yang lain.}
\end{itemize}

Pada bab ini saya menerjemahkan program Card dari bab sebelumnya dari procedural ke gaya orientasi objek. Anda dapat mengunduh source code dari bab ini dari \url{http://thinkapjava.com/code/Card3.java}

\section{Method Objek dan Class Method} % 15.2

Ada dua jenis method pada Java yaitu class method dan object method. Class Method di identifikasi dengan kata kunci static pada baris pertamanya. Dan method yang tidak ada kata kunci static adalah object method.

Meskipun kita belum menulis object method, beberapa telah kita panggil. Kapan saja anda memanggil sebuah method pada sebuah objek adalah Object Method. Sebagai contoh charAt dan method lain yang kita panggil dari objek-objek String semua itu adalah juga merupakan object method. 

Apapun yang dapat di tulis sebagai class method dapat juga ditulis sebagai object method begitu juga sebaliknya. Tetapi terkadang lebih natural untuk menggunakan salah satu di antaranya.
Sebagai contoh ini adalah printCard adalah sebagai class method :

\begin{lstlisting}
public static void printCard(Card c) {
System.out.println(ranks[c.rank] + " of " + suits[c.suit]);
}
\end{lstlisting}

berikut di tulis ulang menjadi object method 

\begin{lstlisting}
public void print() {
System.out.println(ranks[rank] + " of " + suits[suit]);
}
\end{lstlisting}

berikut adalah perubahannya :
\begin{enumerate}
	\item Saya menghilangkan kata kunci static
	\item Saya mengubah nama methodnya menjadi lebih idiomatis
	\item Saya menghilangkan parameternya
	\item Didalam object method anda dapat merujuk ke instance variable sebagaimana jika variable tersebut adalah local variable (variable yang di deklarasikan di dalam body method). Jadi saya mengubah c.rank menjadi rank. Dan begitu juga untuk suit.
\end{enumerate}

Berikut adalah bagaimana method tersebut dipanggil :
\begin{lstlisting}
Card card = new Card(1,1);
card.print();
\end{lstlisting}

ketika anda memanggil sebuah method dari sebuah objek, objek tersebut menjadi current object (objek yang sekang sedang aktif) juga bisa disebut this. Didalam method print kata kunci this merujuk pada objek card yang mana sebagai objek dari method tersebut sedang di panggil.

\section{Method toString} %15.3

setiap objek memiliki sebuah method bernama toString yang mana mengembalikan nilai string representasi dari objeknya. Ketika anda mencetak sebuah objek dengan menggunakan method print atau println java memanggil method toString.

Versi standard dari method toString mengembalikan sebuah string yang mengandung tipe object dan sebuah identifier unik(lihat subbab 11.6). ketika anda mendefinisikan sebuah tipe objek baru anda dapat override (memodifikasi) perilaku dasar dengan menuliskan method baru dengan perilaku yang anda inginkan.

Sebagai contoh berikut adalah sebuah method toString untuk class Card
\begin{lstlisting}
public String toString() {
return ranks[rank] + " of " + suits[suit];
}
\end{lstlisting}

tipe data pengembalian nya adalah String dan tidak ada parameter. Anda dapat memanggil method toString dengan cara biasanya :

\begin{lstlisting}
Card card = new Card(1, 1);
String s = card.toString();
\end{lstlisting}

atau anda dapat memanggil nya melalui method println 
\begin{lstlisting}
System.out.println(card);
\end{lstlisting}

\section{method Equal} %15.4

Pada subbab 13.4 kita telah membicarakan tentang dua notasi persamaan : persamaan, artinya bahwa dua variabel merujuk pada objek yang sama, dan ekivalensi, artinya bahwa mereka memiliki nilai yang sama.

operator == untuk menguji persamaan, tetapi tidak ada operator yang dapat menguji ekivalensi, karena arti dari \texttt{"Ekivalensi"} di sini artinya bergantung pada tipe objeknya. Sebagai gantinya, Objek memiliki method yang bernama equals yang mendefinisikan ekivalensi.

Java menyediakan method equals yang menjalankan fungsi yang sebenarnya. Tetapi untuk buatan user, hanya sebatas melakukan persamaan saja, yang mana tidak seperti yang anda inginkan.

Untuk Card kita memiliki sebuah mthod yang mengecek ekivalensi 
\begin{lstlisting}
public static boolean sameCard(Card c1, Card c2) {
return (c1.suit == c2.suit && c1.rank == c2.rank);
}
\end{lstlisting}

jadi yang harus kita lakukan adalah menulis kembali sebuah object method 
\begin{lstlisting}
public boolean equals(Card c2) {
return (suit == c2.suit && rank == c2.rank);
}
\end{lstlisting}

saya menghapus static dan parameter pertama , c1, dan berikut bagaimana method tersebut di panggil
\begin{lstlisting}
Card card = new Card(1, 1);
Card card2 = new Card(1, 1);
System.out.println(card.equals(card2));
\end{lstlisting}

Didalam equals, card adalah current object dan card2 adalah parameternya, c2. Untuk method yang mengoperasikan 2 object dengan tipe yang sama, saya terkadang menggunakan this secara eksplisit dan memanggil parameter that. 
\begin{lstlisting}
public boolean equals(Card that) {
return (this.suit == that.suit && this.rank == that.rank);
}
\end{lstlisting}

saya rasa ini akan lebih mudah di baca.